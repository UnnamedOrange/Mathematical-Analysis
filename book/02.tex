% Licensed under the Creative Commons Attribution Share Alike 4.0 International.
% See the LICENCE file in the repository root for full licence text.

\section{几个常用不等式}

\subsection{三角不等式}

实数 $x$ 的绝对值定义为:
$$
|x| =
\begin{cases}
x & x \ge 0
\\
-x & x < 0
\end{cases}
$$

$\forall x,y \in \R$,有 $-|x| \le x \le |x|, -|y| \le y \le |y|$,相加得:
$$
-(|x| + |y|) \le x + y \le (|x| + |y|)
$$

由上式,根据含绝对值的不等式的等价变形,有:
$$
|x + y| \le (|x| + |y|)
$$

称上式为\emph{三角不等式}。令 $x = x - y$,由三角不等式可得 $|x - y| \ge |x| - |y|$;令 $y = y - x$,由三角不等式可得 $|x - y| \ge |y| - |x|$,于是有:
$$
||x| - |y|| \le |x - y|
$$

\subsection{伯努利不等式}

$$
\forall x \ge -1, n \in \N, \quad (1 + x)^n \ge 1 + nx
$$

称上式为\emph{伯努利不等式}。

当 $x \ge 0$ 时,用二项式定理即可证明。当 $x \ge -1$ 时,用数学归纳法证明。

\begin{proof}[数学归纳法]
	当 $n = 1$ 时,显然成立。

	当 $n \ge 2$ 时,假设在 $k = n - 1$ 时成立,下面证明在 $k = n$ 时也成立。
	$$
	\begin{aligned}
		(1 + x)^k &= (1 + x)(1 + x)^{k - 1}
		\\&\ge
		(1 + x)(1 + (k - 1)x)
		\\&=
		1 + (k - 1)x + x + (k - 1) x^2
		\\&=
		1 + kx + (k - 1) x^2
		\\&\ge
		1 + kx
	\end{aligned}
	$$
	成立。由数学归纳法知,不等式对所有正整数 $n$ 成立。
\end{proof}

\subsection{算术-几何平均不等式}

对任意 $n$ 个非负实数 $x_1, x_2, \ldots, x_n$,有:
$$
\dfrac{x_1 + x_2 + \cdots + x_n}{n} \ge \sqrt[n]{x_1 x_2 \cdots x_n}
$$

称上式为为\emph{算术-几何平均不等式}。

\begin{proof}[数学归纳法]
	(有选择数的自由度,所以挑较小的 $n - 1$ 个。好处是我们可以得到“较小的 $n - 1$ 个数的均值小于等于最大的那个数”这个额外关系)

	当 $n = 1$ 时,$x_1 \ge x_1$,显然成立。

	当 $n \ge 2$ 时,假设当 $k = n - 1$ 时成立,下面证明对 $k = n$ 时也成立。

	不妨设 $x_k$ 最大,则 $x_k \ge \dfrac{x_1 + \cdots + x_{k - 1}}{k - 1} \ge \sqrt[k - 1]{x_1 x_2 \cdots x_{k - 1}}$。设 $y = \dfrac{x_1 + x_2 + \cdots + x_{k - 1}}{k - 1}$,则有:
	$$
	\begin{aligned}
		\left( \dfrac{x_1 + x_2 + \cdots + x_k}{k} \right)^k &= \left( \dfrac{x_1 + x_2 + \cdots + x_k - ky}{k} + y \right)^k
		\\&=
		\left( \dfrac{x_k - y}{k} + y \right)^k
		\\&\ge
		y^k + y^{k - 1} k \dfrac{x_k - y}{k}
		\\&=
		y^{k - 1} x_k
		\\&\ge
		x_1 x_2 \cdots x_k
	\end{aligned}
	$$
	成立。由数学归纳法知,不等式对所有正整数 $n$ 成立。
\end{proof}

\subsection{例题}

设 $a, b > 0$,求证:
\begin{enumerate}
	\item 当 $p > 1$ 时,$a^p + b^p \le (a + b)^p$;
	\item 当 $0 < p < 1$ 时,$a^p + b^p \ge (a + b)^p$。
\end{enumerate}

\begin{proof}
	令 $p = 1 + h \pod{h > 0}$(将分界线从 $1$ 变为 $0$,从无符号差距到有符号差距),有:
	$$
	\begin{aligned}&
		(a + b)^p
		\\=~&
		(a + b)(a + b)^{p - 1}
		\\=~&
		a (a + b)^{p - 1} + b (a + b)^{p - 1}
		\\\ge~&
		a^p + b^p
	\end{aligned}
	$$

	令 $p = 1 - h \pod{h > 0}$,有:
	$$
	\begin{aligned}&
		(a + b)^p
		\\=~&
		a (a + b)^{p - 1} + b (a + b)^{p - 1}
		\\\le~&
		a^p + b^p
	\end{aligned}
	$$
\end{proof}

\subsection{常用记号}

\begin{definition}{无穷区间}
	$(-\infty, b), (-\infty, b], (a, +\infty), [a, \infty), (-\infty, +\infty)$。
\end{definition}

\begin{definition}{邻域}
	$(a - \varepsilon, a + \varepsilon) \pod{\varepsilon > 0}$,称之为 $a$ 的一个 \emph{$\varepsilon$ 邻域},记为 $U(a, \varepsilon)$。
\end{definition}

\begin{definition}{去心邻域}
	$(a - \varepsilon, a + \varepsilon) \backslash \{ a \} \pod{\varepsilon > 0}$,称之为 $a$ 的一个去心邻域,记为 $U_0(a, \varepsilon)$。
\end{definition}

\section{函数的定义}

\begin{definition}{函数}
	对于给定的集合 $X \subseteq \R$,如果存在某种对应法则 $f$,使得对 $X$ 中的每一个数 $x$,在 $\mathbb R$ 中存在唯一的数 $y$ 与之对应,则称对应法则 $f$ 为从 $X$ 到 $\mathbb R$ 的一个\emph{函数}。记作:
	$$
	\begin{aligned}
		f:~&X \rightarrow \R
		\\&
		x \mapsto y = f(x)
	\end{aligned}
	$$

	(需要指明 $f$、$X$、$\R$,才是一个函数,光说 $f$ 不是函数;$\R$ 不是值域,只是说从 $X$ 到 $\R$)
\end{definition}

函数的 3 个基本要素是:定义域、值域、对应法则。当这三者相等时,函数相等。当然,若定义域和对应法则相同,值域也相同,因此只用比较定义域和对应法则。

\begin{definition}{值}
	$y$ 称为 $f$ 在点 $x$ 的\emph{值}。
\end{definition}

\begin{definition}{定义域}
	$X$ 称为函数 $f$ 的\emph{定义域}。
\end{definition}

\begin{definition}{值域}
	数集 $\{f(x): x \in X\}$ 称为函数 $f$ 的\emph{值域},记为 $f(X)$。
\end{definition}

\begin{definition}{自变量,因变量}
	$x$ 称作\emph{自变量},$y$ 称作\emph{因变量}。
\end{definition}

\begin{definition}{函数的图像}
	设 $y = f(x)$ 是定义在 $X$ 上的函数。称平面点集 $T = \{(x, y): y = f(x), x \in X\}$ 为\emph{函数的图像}。
\end{definition}

\subsection{基本初等函数}

\begin{enumerate}
	\item 常值函数 $y = C$;
	\item 幂函数 $y = x^\alpha \pod{\alpha > 0}$;
	\item 指数函数 $y = a^x \pod{a > 0}$;
	\item 对数函数 $y = \log_a x \pod{a > 0, a \ne 1}$;
	\item 三角函数 $y = \sin x$、$y = \cos x$、$y = \tan x$、$y = \cot x$;
	\item 反三角函数 $y = \arcsin x$、$y = \arccos x$、$y = \arctan x$、$y = \arccot x$。
\end{enumerate}

以上函数统称为\emph{基本初等函数}。

\subsection{符号函数}

$$
y = \sgn x =
\begin{cases}
	1, & x > 0
	\\
	0, & x = 0
	\\
	-1, & x < 0
\end{cases}
$$

\subsection{狄利克雷(Dirichlet)函数}

$$
y = D(x) =
\begin{cases}
	1, & x \in \Q
	\\
	0, & x \in \R \backslash \Q
\end{cases}
$$

\subsection{高斯(Gauss)取整函数}

$y = [x]$ 表示不超过 $x$ 的最大整数。若 $n \le x < n + 1$,则 $[x] = n$。

\subsection{黎曼(Riemann)函数}

$$
y = R(x) =
\begin{cases}
	\dfrac{1}{p}, & x = \dfrac{q}{p} \in (0, 1), \gcd(p, q) = 1
	\\
	0, & x \in[0, 1] \backslash \Q
	\\
	1, & x = 0 ~\text{或}~ 1
\end{cases}
$$

\subsection{特征函数}

$$
y = \chi_E(x) =
\begin{cases}
	1, & x \in E
	\\
	0, & x \not \in E
\end{cases}
$$

\section{构造新函数}

\subsection{函数的四则运算}

设 $y = f_j(x), x \in X_j \subset \R \pod{j = 1, 2}$ 为两个已知函数,且 $X = X_1 \cap X_2 \ne \varnothing$(即:新函数的定义域是原定义域的交集),则可定义:

$$
\begin{gathered}
	(f_1 \pm f_2)(x) = f_1(x) \pm f_2(x) \pod{x \in X}
	\\
	(f_1 f_2)(x) = f_1(x)f_2(x) \pod{x \in X}
	\\
	\dfrac{f_1}{f_2}(x) = \dfrac{f_1(x)}{f_2(x)} \pod{f_2(x) \ne 0} \pod{x \in X}
\end{gathered}
$$

\subsection{函数的限制和延拓}

\begin{definition}{限制,延拓}
	设函数 $f(x), x \in X_1$ 和 $g(x), x \in X_2$ 满足 $X_1 \subset X_2$,且 $f(x) \equiv g(x), \forall x \in X_1$,则称 $f(x)$ 是 $g(x)$ 在 $X_1$ 上的\emph{限制},而 $g(x)$ 是 $f(x)$ 在 $X_2$ 上的\emph{延拓}。(在共享的一段上相同即可)
\end{definition}

对于两个函数而言,只有当他们的定义域及对应关系完全一样时,才能说它们是相等的。而函数的表达式不同不一定代表对应关系不同,例如 $|x| = x \sgn x$。

\subsection{函数的复合}

\begin{definition}{复合函数,内函数,外函数}
	设 $y = f_j(x), x \in X_j \subset \R \pod{j = 1, 2}$ 为两个函数,若 $Y_1 = f_1(X_1) \subseteq X_2$,则定义在 $X_1$ 上的函数 $y = f_2(f_1(x))$ 称为 $f_1$ 和 $f_2$ 的\emph{复合函数},记作 $f_2 \circ f_1: X_1 \rightarrow \R$。通常称 $f_1$ 为该复合函数的\emph{内函数},$f_2$ 为\emph{外函数}。(内函数的值域包含于外函数的定义域)
\end{definition}

一般来说,函数复合不具有交换律。ß

\subsection{例题}

设 $f(x) = \dfrac{x}{\sqrt{1 + x^2}}$,求 $\underset{n ~\text{个}}{\underbrace{f \circ \cdots \circ f}}(x)$ 的表达式。

\begin{solve}
	$$
	(f \circ f)(x) = \dfrac{\dfrac{x}{\sqrt{1 + x^2}}}{\sqrt{1 + \dfrac{x^2}{1 + x^2}}} = \dfrac{x}{\sqrt{1 + 2x^2}}
	$$

	猜想 $(\underset{n ~\text{个}}{\underbrace{f \circ f \circ \cdots \circ f}})(x) = \dfrac{x}{\sqrt{1 + nx^2}}$。假设 $k = n - 1$ 时成立,下面证明 $k = n$ 时也成立:
	$$
	\begin{aligned}
		(\underset{k ~\text{个}}{\underbrace{f \circ f \circ \cdots \circ f}})(x) &= f \left( \dfrac{x}{\sqrt{1 + (k - 1) x^2}} \right)
		\\&=\cdots
		\\&=
		\dfrac{x}{\sqrt{1 + kx^2}}
	\end{aligned}
	$$
	成立。由数学归纳法知 $(\underset{n ~\text{个}}{\underbrace{f \circ f \circ \cdots \circ f}})(x) = \dfrac{x}{\sqrt{1 + nx^2}}$。
\end{solve}