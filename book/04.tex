% Licensed under the Creative Commons Attribution Share Alike 4.0 International.
% See the LICENCE file in the repository root for full licence text.

\section{函数的奇偶性}

\begin{definition}{奇函数,偶函数}
	设 $y = f(x)$ 是在 $X$ 上的函数,且 $X$ 关于原点对称,即 $\forall x \in X, -x \in X$。

	\begin{enumerate}
		\item 若 $\forall x \in X, f(x) = -f(x)$,则 $f(x)$ 是 $X$ 上的\emph{奇函数}(图像关于原点对称);
		\item 若 $\forall x \in X, f(x) = -f(x)$,则 $f(x)$ 是 $X$ 上的\emph{偶函数}(图像关于 $y$ 轴对称)。
	\end{enumerate}
\end{definition}

\subsection{例题}

设 $y = f(x)$ 是 $X$ 上的奇函数且存在反函数。求证:它的反函数也是奇函数。

\begin{proof}
	由 $f(x)$ 是奇函数,得 $\forall x \in X, f(-x) = -f(x)$,则若与 $f^{-1}$ 复合,得:$f^{-1}(f(-x)) = f^{-1}(-f(x))$。则 $\text{L.H.S} = -x = -f^{-1}(f(x))$,则 $\forall y \in f(X), -f^{-1}(y) = f^{-1}(-y)$,则反函数也为奇函数。
\end{proof}

\section{序列的极限}

\begin{definition}{序列,通项}
	\emph{序列}是正整数集 $\N$ 到实数集 $\R$ 的一个函数 $f: \N \rightarrow \R$。

	将序列 $x_1 = f(1), x_2 = f(2), \ldots$ 记为 $\{x_n\}$,其中 $x_n$ 称为\emph{通项}。
\end{definition}

有时也把一个序列的所有数作成的集合记为 $\{x_n\} = f(\N)$。通过在前面加“集合”和“序列”来区分。

\begin{definition}{序列收敛,序列极限}
	设序列 $\{x_n\}$。若 $\exists a \in \R, \st \forall \varepsilon > 0, \exists N \in \N$,当 $n > N$ 时,有 $|x_n - a| < \varepsilon$。则称\emph{序列收敛},而 $a$ 为\emph{序列极限}(或称序列 $x_n$ 是\emph{收敛于 $a$} 的)。记为 $\lim\limits_{b \rightarrow \infty} x_n = a$,或 $x_n \rightarrow a\pod{n \rightarrow \infty}$。

	序列收敛的几何意义是 $\forall \varepsilon > 0$,$U(a, \varepsilon)$ 外只有 $\{x_n\}$ 的有限项。
\end{definition}

\begin{definition}{发散}
	若不存在使得序列收敛的 $a$,则 $\{x_n\}$ \emph{发散}。

	用肯定的语气描述序列发散:$\forall a \in \R, \exists \varepsilon_0 > 0, \st \forall N \in \N$,总 $\exists n_0 > N$,满足 $|x_{n_0} - a| > \varepsilon_0$,则 $\{x_0\}$ 是\emph{发散序列}。
\end{definition}

\section{无穷小量}

\begin{definition}{无穷小量}
	设 $\{x_n\}$ 是一个序列。若 $x_n \rightarrow 0 \pod{n \rightarrow \infty}$,则称序列 $\{x_n\}$ 为\emph{无穷小量},记为 $x_n = o(1) \pod{n \rightarrow \infty}$。
\end{definition}

\begin{theorem}
	设 $\{x_n\}$ 是一个序列。

	\begin{enumerate}
		\item $\{x_n\}$ 是无穷小量的充分必要条件是 $\{|x_n|\}$ 是无穷小量;
		\item 若 $\{x_n\}$ 是无穷小量,$M$ 是一个常数,则 $\{M x_n\}$ 是无穷小量;
		\item $\lim\limits_{n \rightarrow \infty} x_n = a$ 的充分必要条件是 $\{x_n - a\}$ 是无穷小量。
	\end{enumerate}
\end{theorem}

\section{无穷大量}

\begin{definition}{正无穷大量,负无穷大量,无穷大量}
	设 $\{x_n\}$ 为一序列。

	\begin{enumerate}
		\item 若 $\forall M > 0, \exists N \in \N$,当 $n > N$ 时,$x_n > M$,则称 $\{x_n\}$ 是\emph{正无穷大量}(有时称极限为 $+\infty$ 并记为 $\lim\limits_{n \rightarrow \infty} = +\infty$);
		\item 若 $\forall M > 0, \exists N \in \N$,当 $n > N$ 时,$x_n < -M$,则称 $\{x_n\}$ 是\emph{负无穷大量}(有时称极限为 $-\infty$ 或 $\lim\limits_{n \rightarrow \infty} = - \infty$);
		\item 若 $\{|x_n|\}$ 是正无穷大量,则 $\{x_n\}$ 是\emph{无穷大量}(有时称 $\{x_n\}$ 的极限为 $\infty$,记为 $\lim\limits_{n \rightarrow \infty} = \infty$)。
	\end{enumerate}
\end{definition}

注意:有穷极限 $a$ 被称为收敛到 $a$;无穷极限 $+\infty$、$-\infty$、$\infty$ 被称为发散到 $+\infty$、$-\infty$、$\infty$;有些序列既不收敛,也不是无穷大量,如 $x_n = (-1)^n$。凡提到极限存在,均指有穷极限的情形;若包括无穷极限时,则说其\emph{广义极限}存在,此时也说序列是\emph{广义收敛的}。

\subsection{例题}

设 $\lim\limits_{n \rightarrow \infty} x_n = a$,试证 $\lim\limits_{n \rightarrow \infty} \dfrac{x_1 + \cdots + x_n}{n} = a$,这里 $a$ 可以是有限实数,$+\infty$ 或 $-\infty$。

\begin{proof}
	先证 $a$ 是有限实数的情形。$\forall \varepsilon > 0$,由于 $\lim\limits_{n \rightarrow \infty} x_n = a$,故 $\exists N_1$,使得当 $n > N_1$ 时,有 $|x_n - a| < \dfrac{\varepsilon}{2}$。

	对于取定的正整数 $N_1$,由于 $|x_1 - a| + |x_2 - a| + \cdots + |x_{N_1} - a|$ 是一个不随 $n$ 的变化而变化的常数,因此有:
	$$
	\lim\limits_{n \rightarrow \infty} \dfrac{|x_1 - a| + |x_2 - a| + \cdots + |x_{N_1} - a|}{n} = 0
	$$

	于是,又存在正整数 $N_2 > 0$,当 $n > N_2$ 时,有:
	$$
	\dfrac{|x_1 - a| + |x_2 - a| + \cdots + |x_{N_1} - a|}{n} < \dfrac{\varepsilon}{2}
	$$

	现在取 $N = \max\{N_1, N_2\}$,则当 $n > N$ 时,有:
	$$
	\begin{aligned}
		\left| \dfrac{x_1 + \cdots + x_n}{n} - a \right| &\le \dfrac{|x_1 - a| + |x_2 - a| + \cdots + |x_n - a|}{n}
		\\&=
		\dfrac{|x_1 - a| + \cdots + |x_{N_1} - a|}{n} + \dfrac{|x_{N_1 + 1} - a| + \cdots + |x_n - a|}{n}
		\\&\le
		\dfrac{\varepsilon}{2} + \dfrac{n - N_1}{2n} \varepsilon < \varepsilon
	\end{aligned}
	$$
	从而有 $\lim\limits_{n \rightarrow \infty} \dfrac{x_1 + \cdots + x_n}{n} = a$。

	下证 $a = +\infty$ 的情形。$\forall M > 0$,由 $\lim\limits_{n \rightarrow \infty} x_n = +\infty$ 知,$\exists N_1$,当 $n > N_1$ 时,由 $x_n > 2M$,且 $x_1 + x_2 + \cdots + x_{N_1} > 0$,从而有:
	$$
	\dfrac{x_1 + \cdots + x_n}{n} > \dfrac{x_{N_1 + 1} + \cdots + x_n}{n} > \dfrac{2(n - N_1)}{n} M
	$$

	现在取 $N = 2N_1$,则当 $n > N$ 时,有:
	$$
	\dfrac{n - N_1}{n} = 1 - \dfrac{N_1}{n} > \dfrac{1}{2}
	$$

	从而有:
	$$
	\dfrac{x_1 + \cdots + x_n}{n} > \dfrac{2(n - N_1)}{n} M > M
	$$

	这就证明了 $\lim\limits_{n \rightarrow \infty} \dfrac{x_1 + \cdots + x_n}{n} = +\infty$。

	$a = -\infty$ 的情形证明类似。
\end{proof}