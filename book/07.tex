% Licensed under the Creative Commons Attribution Share Alike 4.0 International.
% See the LICENCE file in the repository root for full licence text.

\section{闭区间套定理}

\begin{theorem}[闭区间套定理]
	设 $\{[a_n, b_n]\}$ 是一列闭区间,并满足:

	\begin{enumerate}
		\item $[a_n, b_n] \supseteq [a_{n + 1}, b_{n + 1}]$;
		\item $\lim\limits_{n \rightarrow \infty} (b_n - a_n) = 0$。
	\end{enumerate}

	则存在唯一的一点 $c \in \R$,使得 $c \in [a_n, b_n], n = 1, 2, \ldots$,即:
	$$
	\{c\} = \bigcap_{n = 1}^{\infty} [a_n, b_n]
	$$
\end{theorem}

\begin{proof}
	$\forall n \in \N, a_n \le a_{n + 1} \le b_{n + 1} \le b_n$,即 $\{a_n\}$ 单调上升,$\{b_n\}$ 单调下降,且 $a_n \le b_1$,$b_n \ge a_1$。由单调收敛原理,可设 $\{a_n\}$ 收敛于 $a$,$\{b_n\}$ 收敛于 $b$,即:$\lim\limits_{n \rightarrow \infty} a_n = a$,$\lim\limits_{n \rightarrow \infty} b_n = b$。

	又有 $\lim\limits_{n \rightarrow \infty} b_n - \lim\limits_{n \rightarrow \infty} a_n = 0$,于是 $a = b$。由 $a_n \le a$,$b \le b_n$ 可知,存在 $c = a = b$,使得 $a_n \le c \le b_n$。存在性证毕。

	下证唯一性。若另有 $d$ 满足 $a_n \le d \le b_n$,由夹逼收敛原理,得 $d = a = b = c$,故 $c$ 是唯一存在的。
\end{proof}

注:

\begin{enumerate}
	\item 若区间是开的,则定理的结论不一定成立。例如,对于开区间列 $(0, \dfrac{1}{n})$,其交集为空集(对任意数,总能找到一个 $n$ 使得它在区间外)。
	\item 开区间列的条件 2 改为 $a_n < a_{n + 1} < b_{n + 1} < b_n$,则仍成立。
	\item “闭区间套”可松弛为两两有公共点。
\end{enumerate}

\section{有限覆盖定理}

\begin{definition}{覆盖}
	设 $A \subseteq \R$,$\{E_\lambda\}_{\lambda \in \Lambda}$ 是 $\R$ 的一族子集组成的集合,其中 $\Lambda$ 是一个指标集(下标集,不一定是可数的那种,比如实数也行)。若 $A \subseteq \bigcup\limits_{\lambda \in \Lambda} E_\lambda$,则称 $\{E_\lambda\}_{\lambda \in \Lambda}$ 是 $A$ 的一个\emph{覆盖}(这些集合的并把 $A$ 整个盖住了)。
\end{definition}

\begin{definition}{开覆盖}
	若 $\{E_\lambda\}_{\lambda \in \Lambda}$ 是 $A$ 的一个覆盖,而且对每个 $\lambda \in \Lambda$,而且对每个 $\lambda \in \Lambda$,$E_\lambda$ 均是一个开区间,则称 $\{E_\lambda\}_{\lambda \in \Lambda}$ 是一个\emph{开覆盖}。
\end{definition}

\begin{definition}{有限覆盖}
	若 $\{E_\lambda\}_{\lambda \in \Lambda}$ 是 $A$ 的一个覆盖,而且 $\Lambda$ 的元素只有有限多个,则称 $\{E_\lambda\}_{\lambda \in \Lambda}$ 是 $A$ 的一个\emph{有限覆盖}。
\end{definition}

\begin{theorem}{有限覆盖定理}
	设 $[a, b]$ 是一个\textbf{闭区间},$\{E_\lambda\}_{\lambda \in \Lambda}$ 是 $[a, b]$ 的任意一个开覆盖,则必存在 $\{E_\lambda\}_{\lambda \in \Lambda}$ 一个子集构成 $[a, b]$ 的一个有限覆盖,即在 $\{E_\lambda\}_{\lambda \in \Lambda}$ 中必有有限个开区间 $E_1, E_2, \cdots, E_N$,使得:
	$$
	[a, b] \subseteq \bigcup\limits_{j = 1}^N E_j
	$$
\end{theorem}

\begin{proof}[反证法]
	假设不存在 $\{E_\lambda\}_{\lambda \in \Lambda}$ 的一个子集构成 $[a, b]$ 的一个有限覆盖,则不存在 $\{E_\lambda\}_{\lambda \in \Lambda}$ 的一个子集构成 $\left[ a, \dfrac{a + b}{2} \right]$ 或 $\left[ \dfrac{a + b}{2}, b \right]$ 的一个有限覆盖。对任意一个这样的子区间 $[a_n, b_n]$,我们进行同样的询问,则对 $[a_n, b_n]$ 有:

	\begin{enumerate}
		\item $[a_{n + 1}, b_{n + 1}] \subset [a_n, b_n]$;
		\item $(b_n - a_n) \rightarrow 0 \pod{n \rightarrow \infty}$;
		\item 它不能被 $\{E_\lambda\}_{\lambda \in \Lambda}$ 中的有限多个开区间所覆盖。
	\end{enumerate}

	由闭区间套原理,存在唯一的 $\xi$,使得 $\xi \in [a_n, b_n] \pod{\forall n \in \N}$,则 $\exists (c, d) \in \{E_\lambda\}_{\lambda \in \Lambda}, \st c < \xi < d$,则 $\exists N \in \N, \st \forall n > N, c < a_n \le \xi \le b_n < d$,矛盾。
\end{proof}

注:

\begin{enumerate}
	\item 不能将被盖区间从闭区间改为开区间。如 $(\frac{1}{n}, \frac{2}{n})$ 是 $(0, 1)$ 的一个开覆盖,但没有有限覆盖。
	\item 该定理是说,对于闭区间的任意一个开覆盖必可从中找出有限多个开区间就可将该闭区间覆盖,而并不是说,任意一个闭区间都可被有限多个开区间覆盖。若是这样,$(a - 1, b + 1)$ 总是 $[a, b]$ 的一个有限开覆盖,就没意义了。
	\item 应用中:局部(无限多)$\ce{->[\text{不能推出}]}$ 整体;局部(无限多) $\ce{->[\text{有限覆盖定理}]}$ 局部(有限多)→ 整体。
\end{enumerate}

\section{聚点原理}

\begin{definition}{聚点}
	设 $E$ 是 $\R$ 的一个子集。若 $x_0 \in \R$($x_0$ 不一定属于 $E$)满足:对 $\forall \delta > 0$,有 $U_0(x_0, \delta) \cap E \ne \varnothing$,则称 $x_0$ 是 $E$ 的一个\emph{聚点}。
\end{definition}

注:

\begin{enumerate}
	\item $x_0$ 是 $E$ 的聚点与 $x_0$ 是否属于 $E$ 无关(例:$x_0 = 0$,$E = (0, 1)$)。
	\item 由聚点的定义容易证明如下三个命题等价:
	\begin{enumerate}
		\item $x_0$ 是 $E$ 的聚点。
		\item $\forall \delta > 0$,在 $U(x_0, \delta)$ 中有 $E$ 的无穷多个点。
		\item 存在 $E$ 中互异的点组成的序列 $\{ x_n \}$,使得 $\lim\limits_{n \rightarrow \infty} x_n = x_0$(例:无限逼近)。
	\end{enumerate}
\end{enumerate}

\begin{definition}{孤立点}
	若 $x_0 \in E$,但不是 $E$ 的聚点,则称 $x_0$ 是 $E$ 的\emph{孤立点}。

	使用符号语言表示:$\exists \delta > 0, \st U(x_0, \delta) \cap E = \{x_0\}$。
\end{definition}

聚点的例子:

\begin{itemize}
	\item $E = \left\{1, \dfrac{1}{2}, \dfrac{1}{3}, \cdots, \dfrac{1}{n}, \cdots \right\}$,$0$ 是聚点,$E$ 中的每个元素都是孤立点。
	\item $E = [0, 1] \cap \Q$,$[0, 1]$ 中的每个\textbf{实数}都是聚点。
\end{itemize}

\begin{theorem}[聚点原理]
	$\R$ 中的任何一个有界无穷子集至少有一个聚点。
\end{theorem}

\begin{proof}[使用有限覆盖定理]
	设 $E$ 为 $\R$ 中的一个有界无穷子集,不妨设 $E \subseteq [a, b]$。

	假设没有聚点,则 $\forall x \in E, \exists \sigma_x > 0$,使得 $U(x, \sigma_x)$ 中只存在一个 $E$ 中的数。

	显然 $\{U(x, \sigma_x): x \in [a, b]\}$ 是一个 $[a, b]$ 的开覆盖。由有限覆盖定理,$[a, b]$ 能被 $\{U(x, \sigma_x): x \in [a, b]\}$ 的一个有限子集覆盖,于是 $E$ 也被它覆盖,而每个 $U(x, \sigma_x)$ 中至多只有一个 $E$ 中的元素,说明 $E$ 不是无穷的,矛盾。
\end{proof}

\begin{proof}[使用闭区间套定理]
	设 $E \subseteq [a, b]$,则 $E \cap \left[ a, \dfrac{a + b}{2} \right]$ 和 $E \cap \left[ \dfrac{a + b}{2}, b \right]$ 中至少有一个是有界无穷数集,任取一个,记为 $[a_1, b_1]$,如此重复,得到 $[a_n, b_n]$。由闭区间套定理,存在一个唯一的 $\xi$,使得 $a_n \le \xi \le b_n \pod{\xi \in E}$,则 $\exists N_0 \in \N, \st \forall n > N_0, U_0(\xi, b_n - a_n) \cap E \ne \varnothing$,即 $\xi$ 是聚点。
\end{proof}

\section{BW 定理}

\begin{definition}{子序列}
	设 $\{x_n\}$ 是一个序列,则由该序列的一部分元素按原来的顺序构成的序列 $\{x_{n_k}\}$ 称为是 $\{x_n\}$ 的一个\emph{子序列}。
\end{definition}

下面给出两个有关子序列的证明显然的定理。

\begin{theorem}
	设 $\lim\limits_{x \rightarrow \infty} x_n = a$,则对 $\{x_n\}$ 的任意子序列 $\{x_{n_k}\}$,都有 $\lim\limits_{k \rightarrow \infty} x_{n_k} = a$。
\end{theorem}

\begin{theorem}
	若在一个序列 $\{x_n\}$ 中可以找到两个收敛的子序列 $\{x_{n_k}\}$ 和 $\{x_{n'_k}\}$,使得他们的极限不同,则该序列必然发散。
\end{theorem}

\begin{theorem}[BW 定理]
	任何有界序列必有收敛的子序列。
\end{theorem}

\begin{proof}[使用聚点原理]
	设 $\{x_n\}$ 是一有界序列,若 $\{x_n\}$ 只由有限多个数所组成,则它必有无穷多项等于同一个数,此时显然成立。否则,$\{x_n\}$ 是一个有界无穷集合,由聚点原理,它必存在一个聚点,设为 $a$。

	下面利用这个聚点构造收敛的子序列。取 $x_{n_1} \in U(a, 1) \cap \{x_1, \ldots\}$,$x_{n_2} \in U \left( a, \dfrac{1}{2} \right) \cap \{x_{n_1 + 1}, \ldots\}$,……以此类推,就找到了一个子列满足 $x_{n_k} \in U \left( a, \dfrac{1}{k} \right)$,从而 $\lim\limits_{k \rightarrow \infty} x_{n_k} = a$。
\end{proof}