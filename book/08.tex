% Licensed under the Creative Commons Attribution Share Alike 4.0 International.
% See the LICENCE file in the repository root for full licence text.

\section{柯西收敛准则}

\begin{definition}{柯西序列}
	设 $\{x_n\}$ 是一个序列,若 $\forall \varepsilon > 0, \exists N$,当 $n, m > N$ 时,有 $|x_n - x_m| < \varepsilon$,则称 $\{x_n\}$ 是一个\emph{柯西序列}。
\end{definition}

\begin{theorem}[柯西收敛准则]
	序列 $\{x_n\}$ 收敛的充分必要条件是它是一个柯西序列。
\end{theorem}

\begin{proof}[首先找到柯西序列的极限]
	必要性显然。下面证充分性。

	设 $\{x_n\}$ 是一个柯西序列,对 $\varepsilon_0 = 1$,存在一个 $N$,当 $n \ge N + 1 > N$ 时,有 $|x_n - x_{N + 1}| < 1$。于是可知 $\{x_n\}$ 是有界的,则由 BW 定理,存在 $\{x_n\}$ 的一个收敛子列 $\{x_{n_k}\}$,记 $a = \lim\limits_{k \rightarrow \infty} x_{n_k}$。

	下面证明 $\lim\limits_{n \rightarrow \infty} x_n = a$。由 $\forall \varepsilon > 0, a = \lim\limits_{k \rightarrow \infty} x_{n_k}$ 知,存在一个 $K$,当 $k \ge K$ 时,有 $|x_{n_k} - a| < \dfrac{\varepsilon}{2}$。再由 $\{x_n\}$ 是柯西序列知,存在一个 $N$,当 $n, m > N$ 时,有 $|x_n - x_m| < \dfrac{\varepsilon}{2}$。

	这样,取 $k > K$,并且使得 $n_k > N$,则当 $n > N$ 时,有:
	$$
	|x_n - a| \le |x_n - x_{n_k}| + |x_{n_k} - a| < \varepsilon
	$$
	即 $\lim\limits_{n \rightarrow \infty} x_n = a$。
\end{proof}

\begin{theorem}[压缩映照原理]
	设 $f(x)$ 在 $[a, b]$ 上有定义,$f([a, b]) \subset [a, b]$,且满足 $\forall x, y \in [a, b], |f(x) - f(y)| \le q |x - y|$,其中 $0 < q < 1$。则存在唯一的 $c \in [a, b]$,使得 $f(c) = c$。
\end{theorem}

\begin{proof}
	任取 $x_0 \in [a, b]$,由条件 $f([a, b]) \subset [a, b]$,可定义 $x_n = f(x_{n - 1})$。则:
	$$
	|x_{n + 1} - x_n| = |f(x_n) - f(x_{n - 1})| \le q |x_n - x_{n - 1}| \le q^n |x_1 - x_0|
	$$

	则:
	$$
	|x_{n + p} - x_n| \le \sum_{k = 1}^p |x_{n + k} - x_{n + k - 1}| \le \sum_{k = 1}^p q^{n + k - 1} |x_1 - x_0| = \dfrac{1 - q^p}{1 - q} q^n |x_1 - x_0| < \dfrac{q^n}{1 - q} |x_1 - x_0| = L q^n
	$$

	则 $\{x_n\}$ 是一个柯西序列,于是它收敛,设它收敛于 $c$。又根据已知条件有 $|f(x_n) - f(c)| \le q|x_{n - 1} - c|$,于是 $f(x_n)$ 也收敛,并且收敛于 $f(c)$。现在在等式 $x_n = f(x_{n - 1})$ 的两边取极限,即得 $c = f(c)$。存在性证毕,下证唯一性。

	假设另有 $c'$ 也满足条件,则应有 $|c - c'| = |f(c) - f(c')| \le q|c - c'|$。矛盾,故唯一性得证。
\end{proof}

\section{序列的上、下极限}

\begin{definition}{下极限,上极限}
	设 $\{x_n\}$ 是有界序列。令 $l_n = \inf \{x_n, x_{n + 1}, x_{n + 2}, \ldots\}$、$h_n = \sup \{x_n, x_{n + 1}, x_{n + 2}, \ldots\}$,则有:
	$$
	l_1 \le l_2 \le \cdots \le l_n \le \cdots \le h_n \le \cdots \le h_2 \le h_1
	$$
	即 $\{l_n\}$ 和 $\{h_n\}$ 是单调有界序列。令 $l = \sup\limits_n \{l_n\}$,$h = \inf\limits_n \{h_n\}$。
	$$
	\begin{gathered}
		l = \sup\limits_n \inf\limits_k \{x_{n + k}\}
		\\
		h = \inf\limits_n \sup\limits_k \{x_{n + k}\}
	\end{gathered}
	$$
	则 $l$ 与 $h$ 分别称为 $\{x_n\}$ 的\emph{下极限}和\emph{上极限},记为 $l = \llim\limits_{n \rightarrow \infty} x_n$,$r = \ulim\limits_{n \rightarrow \infty} x_n$。有时下极限和上极限也记为 $\liminf\limits_{n \to \infty} x_n$ 和 $\limsup\limits_{n \to \infty} x_n$。
\end{definition}

此外,对于无界序列,我们规定:

\begin{enumerate}
	\item 若 $\{x_n\}$ 无上界,则 $\ulim\limits_{n \rightarrow \infty} x_n = + \infty$。注意此时必有 $h_n = +\infty, \forall n \in \N$。
	\item 若 $\{x_n\}$ 无下界,则 $\llim\limits_{n \rightarrow \infty} x_n = -\infty$。注意此时必有 $h_n = -\infty, \forall n \in \N$。
	\item 若 $\{x_n\}$ 有下界但无上界,则 $\{l_n\}$ 有定义,故可定义: $\llim\limits_{n \rightarrow \infty} x_n = \sup\limits_{n} \{l_n\}$,但需注意的是,此时 $\sup\limits_n \{l_n\}$ 可能为 $+\infty$。
	\item 若 $\{x_n\}$ 有上界但无下界,则 $\{h_n\}$ 有定义,故可定义:$\ulim\limits_{n \rightarrow \infty} x_n = \inf\limits_n\{h_n\}$,但需注意的是,此时 $\inf\limits_n\{h_n\}$ 可能为 $-\infty$。
\end{enumerate}

加上以上四条规定,则对任意的序列 $\{x_n\}$,$\ulim\limits_{n \rightarrow \infty} x_n$ 和 $\llim\limits_{n \rightarrow \infty} x_n$ 都有明确的定义,而且恒有 $\llim\limits_{n \rightarrow \infty} x_n \le \ulim\limits_{n \rightarrow \infty} x_n$。

\begin{theorem}
	设 $\{x_n\}$ 是一有界序列,$h$ 是一实数,则下列三个命题等价:

	\begin{enumerate}
		\item $h$ 是 $\{x_n\}$ 的上极限;
		\item $\forall \varepsilon > 0, \exists N$,当 $n > N$ 时,有 $x_n < h + \varepsilon$,而且对 $\forall K, \exists n_k > K$,使得 $x_{n_k} > h - \varepsilon$;
		\item 存在子列 $\{x_{n_k}\}$,使得 $\lim\limits_{k \rightarrow \infty} x_{n_k} = h$,而且对任何其他收敛子列 $\{x_{n'_k}\}$,有 $\lim\limits_{k \rightarrow \infty} x_{n'_k} \le h$。
	\end{enumerate}
\end{theorem}

\begin{proof}[$1 \rightarrow 2$]
	记 $h_n = \sup \{x_n, x_{n + 1}, \ldots\}$,可知,$\forall \varepsilon > 0, \exists N \in \N, \forall n > N, |h_n - h| < \varepsilon$,即得 $h - \varepsilon < h_n < h + \varepsilon$。而 $x_n \le h_n$,故 $x_n < h + \varepsilon$。

	$\forall K > 0$,当 $n'_k > \max \{N, K\}$ 时,有 $h_{n_k} > h - \varepsilon$。由 $h_{n'_k} = \sup \{ x_{n'_k}, x_{n'_k + 1}, \ldots \}$,可知存在 $n_k \ge n'_k > K$,使得 $x_{n_k} > h - \varepsilon$。
\end{proof}

\begin{proof}[$2 \rightarrow 3$]
	已知对 $\varepsilon = 1$,存在正整数 $n_1$,使得 $h - 1 < x_{n_1} < h + 1$。现假定对 $\varepsilon_k = \dfrac{1}{k} \pod{k \ge 1, k \in \N}$,存在 $n_k \in \N$,满足 $h - \dfrac{1}{k} < x_{n_k} < h + \dfrac{1}{k}$,则存在 $N_{k + 1}$,当 $n > N_{k + 1}$ 时,有 $x_n < h + \dfrac{1}{k + 1}$。令 $K = \max \{N_{k + 1}, n_k\}$,则存在 $n_{k + 1} > K$,使得 $h - \dfrac{1}{k + 1} < x_{n_{k + 1}} < h + \dfrac{1}{k + 1}$。于是可以用以上方法构造子列,该子列即满足 $\lim\limits_{k \to \infty} x_{n_k} = h$。

	现设 $\{x_{n'_k}\}$ 是 $\{x_n\}$ 的一个收敛子列,并设 $\lim\limits_{k \to \infty} x_{n'_k} = h'$,则 $\forall \varepsilon > 0, \exists N \in \N$,使得当 $n'_k > N$ 时,$x_{n'_k} < h + \varepsilon$。由极限的保序性,得 $h' \le h + \varepsilon$,再由 $\varepsilon$ 的任意性,得 $h' \le h$。
\end{proof}

\begin{proof}[$3 \rightarrow 1$]
	设 $h_{n_k}$ 表示 $\sup \{ x_{n_k}, x_{n_k} + 1, \ldots \}$,于是有 $h_{n_k} \ge x_{n_k}$。设 $h'$ 为 $\{ x_n \}$ 的上极限,则可知 $h' \ge h$。下面证明 $h' > h$ 是不可能的,即可证得 $h' = h$。

	假设 $h' > h$,令 $\varepsilon_0 = \dfrac{h' - h}{2} > 0$,则由 BW 定理与极限的保序性,$\forall K \in \N, \exists n_k > K$,使得 $x_{n_k} > h' - \varepsilon$。而这与 $\lim\limits_{k \to \infty} x_{n'_k} \le h$ 矛盾。
\end{proof}

此外,我们不加证明地给出以下定理。

\begin{theorem}
	\begin{enumerate}
		\item 若有界序列 $\{x_n\}$ 由互不相同的数组成,则上极限 $\ulim\limits_{n \rightarrow \infty} x_n$ 是 $\{x_n\}$ 的最大聚点,而下极限 $\llim\limits_{n \rightarrow \infty} x_n$ 是 $\{x_n\}$ 的最小聚点;
		\item $\{x_{n_k}\}$ 是 $\{x_n\}$ 的任一子列,则:$\llim\limits_{n \rightarrow \infty} x_n \le \llim\limits_{n \rightarrow \infty} x_{n_k} \le \ulim\limits_{n \rightarrow \infty} x_{n_k} \le \ulim\limits_{n \rightarrow \infty} x_n$;
		\item $\lim\limits_{n \rightarrow \infty} x_n = a$ 的充分必要条件是 $\llim\limits_{n \rightarrow \infty} x_n = \ulim\limits_{n \rightarrow \infty} x_n = a$,其中 $a$ 可以是有限数、$-\infty$ 或 $+\infty$。
	\end{enumerate}
\end{theorem}

\begin{theorem}
	设 $\{x_n\}$ 和 $\{y_n\}$ 是任意给定的两个有界序列,则有:
	\begin{enumerate}
		\item $x_n \le y_n \pod{n = 1, 2, \cdots}$ 蕴含着 $\llim\limits_{n \rightarrow \infty} x_n \le \llim\limits_{n \rightarrow \infty} y_n$,$\ulim\limits_{n \rightarrow \infty} x_n \le \ulim\limits_{n \rightarrow \infty} y_n$;
		\item $\llim\limits_{n \rightarrow \infty}  (-x_n) = - \ulim\limits_{n \rightarrow \infty} x_n$,$\ulim\limits_{n \rightarrow \infty} (-x_n) = -\llim\limits_{n \rightarrow \infty} x_n$;
		\item $\llim\limits_{n \rightarrow \infty} x_n + \llim\limits_{n \rightarrow \infty} y_n \le \llim\limits_{n \rightarrow \infty} (x_n + y_n) \le \llim\limits_{n \rightarrow \infty} x_n + \ulim\limits_{n \rightarrow \infty} y_n$;

		$\llim\limits_{n \rightarrow \infty} x_n + \ulim\limits_{n \rightarrow \infty} y_n \le \ulim\limits_{n \rightarrow \infty} (x_n + y_n) \le \ulim\limits_{n \rightarrow \infty} x_n + \ulim\limits_{n \rightarrow \infty} y_n$;

		\item 若 $x_n \ge 0, y_n \ge 0, n = 1, 2, \cdots$,则:

		$\llim\limits_{n \rightarrow \infty} x_n \cdot \llim\limits_{n \rightarrow \infty} y_n \le \llim\limits_{n \rightarrow \infty} (x_n \cdot y_n) \le \llim\limits_{n \rightarrow \infty} x_n \cdot \ulim\limits_{n \rightarrow \infty} y_n$,

		$\llim\limits_{n \rightarrow \infty} x_n \cdot \ulim\limits_{n \rightarrow \infty} y_n \le \ulim\limits_{n \rightarrow \infty} (x_n \cdot y_n) \le \ulim\limits_{n \rightarrow \infty} x_n \cdot \ulim\limits_{n \rightarrow \infty} y_n$。
	\end{enumerate}
\end{theorem}