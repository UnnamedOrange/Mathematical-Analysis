% Licensed under the Creative Commons Attribution Share Alike 4.0 International.
% See the LICENCE file in the repository root for full licence text.

\documentclass[UTF8, a4paper]{article}
\usepackage{ctex}
\usepackage{hyperref}
\usepackage{setspace}
\usepackage{listings}
\usepackage{xcolor}
\usepackage{ulem}
\normalem
\usepackage{amsmath}
\usepackage{amsthm}
\usepackage{amsfonts}
\usepackage{amssymb}
\usepackage{booktabs}
\usepackage{graphicx}
\usepackage{cite}
\usepackage{color}
\usepackage{xifthen}
\usepackage{framed}
\usepackage{multicol}
\usepackage[version=4]{mhchem}
\usepackage[top = 0.8in, bottom = 0.8in, left = 0.8in, right = 0.8in]{geometry}

\usepackage{makeidx}
\makeindex

\newcommand \insertsubject {{数学分析}}
\title{\insertsubject}
\author{Orange Lee}

\hypersetup
{
	pdfauthor = Orange Lee,
	pdftitle = \insertsubject,
	pdfsubject = \insertsubject,
	pdfkeywords = \insertsubject,
	colorlinks = true,
	linkcolor = black,
	anchorcolor = black,
	citecolor = black,
	urlcolor = black
}

% index
\newcommand \idx[1] {#1\index{#1}}

% def
\makeatletter
\newcommand \indexer[1]
{%
	\@for\@myvar:={#1}\do{\index{\@myvar}}%
}
\makeatother
\newenvironment{definition}[1]
{
	\definecolor{shadecolor}{RGB}{233, 233, 243}
	\begin{shaded*}
	\indexer{#1}
	\noindent
	\textbf{Def} \emph{#1}:%
}
{
	\end{shaded*}
}

% theorem
\newcounter{cTheorem}
\newenvironment{theorem}[1][]
{
	\definecolor{shadecolor}{RGB}{243, 233, 233}
	\begin{shaded*}
	\noindent
	\textbf{Theorem}
	\ifthenelse{\isempty{#1}}{\stepcounter{cTheorem}\arabic{cTheorem}}{\emph{#1}\index{#1}}:%
}
{
	\end{shaded*}
}

% proof
\renewenvironment{proof}[1][]
{
	\definecolor{shadecolor}{RGB}{237, 237, 237}
	\begin{shaded*}
	{\noindent \textbf{证明}~{#1}}

}
{
	\end{shaded*}
}

% solve
\newenvironment{solve}[1][]
{
	\definecolor{shadecolor}{RGB}{237, 237, 237}
	\begin{shaded*}
	{\noindent \textbf{解}~{#1}}

}
{
	\end{shaded*}
}

% 快捷指令
\newcommand \N {\mathbb N}
\newcommand \Q {\mathbb Q}
\newcommand \R {\mathbb R}
\newcommand \Z {\mathbb Z}
\newcommand \st {\text{s.t.~}}
\newcommand \arccot {\mathop{\mathrm{arccot}}}
\newcommand \sgn {\mathop{\mathrm{sgn}}}
\newcommand \e {\mathrm e}
\newcommand \llim {\mathop{\underline \lim}}
\newcommand \ulim {\mathop{\overline \lim}}